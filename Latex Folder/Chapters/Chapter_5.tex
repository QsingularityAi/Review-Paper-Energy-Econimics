\chapter{Conclusion}
Through our research on the models used in many papers and the methods of model construction, we divided the models into three major categories, structural models, reduced form models and artificial intelligence models. 

The core of structural models is fundamental micro-economic theories about the objectives, constraints, and behaviors of market actors. Establish a mathematical architecture for the relationship between market actors and market behavior. In structural models, we talk about three type: game-theoretical model, optimization model and simulation model. In fact, it is difficult to distinguish between these three models.


The game theory equilibrium model considers two or more optimal agents in the market. These problems can be solved by finding a Nash balance. Because in the Nash equilibrium state, no agent can independently deviate from its strategy and obtain more benefits. The Nash-Cournot model and Stackelberg model simulate the price path of oil under different market structures. Then the Numerical Partial Equilibrium model is equivalent to the combination of the above two models, which allows both Nash-Cournot and Stackelberg behavior in the market. Computational model is equivalent to the optimization of structural model, which can take into account more agents and different factors, but requires strong computational power, so it can only be achieved with the help of high-performance computers.


The optimization model has the perfect ability to predict , choose the best strategy in the whole period. In the optimization model, a single agent looks for the optimal strategy, and simulates the response of other agents to this strategy, and this part of the response is also included in the decision.


Simulation models simulates different strategies applied by agents by taking multiple conditions into consideration. These generated results will give the opportunities to the decision makers and have effect on them especially in the area of political. For example, the oil production country's government measure developmental paths related to the oil output and the price. 

Reduced-form model is less restricted by parameters, and is very suitable for analyzing data quickly and intuitively in order to obtain the relationship between variables.  Compared with the structure model which clearly stipulates the analysis method of oil demand and supply, the simplified model focuses on the analysis of oil price and its sequence of events.  VAR model and SVAR model are the two most widely used reduced-form models. At the same time, the Reduced form model can have a variety of different evolution modes, which can solve a variety of different economic problems.

In the artificial intelligence model, we mainly introduce how the artificial neural network model is constructed, and at each stage of the construction, we introduce the algorithms or parameter adjustment methods that can be selected. This provides a general process for future researchers to use artificial intelligence models to predict oil prices, and what methods can be used in each step.

For the characteristics of these models, structural models provide an explanation for the long-term fluctuations in oil prices and refer to too many model parameters. But sometimes we want to track short-term oil price trends without having to consider market behavior in detail. Therefore, compared with structural models, reduced form models focus on the time changes and statistical relationships of oil prices and other related factors. Generally speaking, it predicts the future oil price advantage in the short term. Artificial neural networks differs from these traditional modeling methods by using parametric methods to build models to solve complex structural and nonlinear and dynamic problems.

