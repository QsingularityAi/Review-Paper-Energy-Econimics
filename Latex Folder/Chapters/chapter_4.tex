\chapter{Discussion}

\section{Structural models}

\subsection{Game-theoretic equilibrium models}
Game theoretic-equilibrium models study the dominating players of the oil market and provide price paths of simulations under several ideal market structures. These models help us to better understand the agents of the markets and their behaviours. However, in recent years, the rise of shale oil has exerted more power in the crude oil market. The equilibrium of the market seems to shift to an other stage. Whether OPEC can still lead the market remains a problem(\cite{ansari2017opec}). As the real oil market is a complex one, it contains more agents than the equilibrium models have.  The models need more computational power when dealing with large volumes of data, in which case, the computational models will take over.

\subsection{Optimization models}
Based on the structure of the optimization model we are able to easily think there are numerous practical challenges to objectify future prices. It's doesn't matter we've got an sufficient amount of information and accurate objective functions. There's uncertainty related element within the development of technological future fuel price fluctuations are typically considered as matters of risk and uncertainties which will be addressed in quantitative risk assessments and sensitivity analyses (\cite{lund2017simulation}). But In some case's optimization model have a really powerful tool to grasp the accurate description of this energy system.

\subsection{Simulation models}
Simulation model takes multiple key factors into consideration and generates various alternative solutions for analyzing and comparing. As the user could use simulation model to calculate and test the resulted performances by combing arbitrarily chosen elements such as cost, budget, market share, and others. It gives users the power and grounds for decision-making (\cite{lund2017simulation}). The limitation in the simulation model is that the model depends more on the empirical evidence rather than economical theory so that the performance of agents' performance and the expected goal may not be reached. Although it demonstrated that agents' behavior might be plausible rather than the optimal.


\section{Reduced form models}
By the research and analysis of the two most widely used and popular reduced-form models. We found that the reduced-form model can effectively solve the parameter dependence problem. Compared with the structural model and the calculation model, the reduced form model has a huge advantage in this regard. At the same time, the reduced-form model can analyze the data more intuitively, and can obtain the relationship between certain variables more quickly. And when we are studying a complex economic process and human behavior, we do not need a lot of model estimation assumptions. However, because the operating mechanism of many reduced-form models, including the VAR model, treats the complex economic process and human behavior as a "black box" as a whole, so they cannot live up to the specific realization mechanism of this set of causality.  Therefore, the use of the SVAR model is a good supplement to this shortcoming and can obtain the relationship between variables.

\section{Artificial intelligence models}
Compared with traditional regression techniques, the advantage of artificial neural networks is that they can solve highly nonlinear complex dynamic problems and does not to know the relationship between the input and output variables in advance. Secondly, it can process continuous data as output or input.

But it also has its problems. The selection and processing of the data set is very complicated, and the accuracy and dimension of the data set are very high. And it failed to generate profits while using non-stationary data. Over-fitting and other issues may also occur during training
