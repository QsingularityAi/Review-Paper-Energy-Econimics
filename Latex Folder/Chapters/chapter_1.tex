\chapter{Introduction}
Oil is an important commodity in the global economy and an important part of the economic development of industrialized and developing countries. In recent years, the world oil market is becoming more and more complex. Political events, extreme weather, financial market speculation, etc. can all affect the oil market and increase the price volatility of the oil market.

This survey focuses on the models used in the oil market. We discusses the method of constructing the model. We also analyze the advantages and disadvantages of each model by applying the model to oil price prediction and provides guidance for future research or oil market modeling.


Based on the characteristics of the model, we divided these models into three kinds: structural models, reduced form models, and artificial intelligence models. These classifications are not to establish a classification method but to facilitate comparison of the characteristics of the model. Although sometimes there is more than one method when modeling.

Section 2 is a historical analysis of oil price developments. In section 3 we introduced the modeling methods of three major models. In section 3.1, we divided structured models into three categories, optimization models, simulation models, and game theory models. Structural models provide an explanation for the long-term fluctuations in oil prices, but sometimes we want to track short-term oil price trends without having to consider market behavior in detail. Structural models usually refer to too many model parameters. Therefore, compared with structural models, in section 3.2 we introduced the reduced form. It focus on statistical relationships of oil prices and the time changes. Since artificial intelligence models have been increasingly used in oil market models in recent years, and this has become a trend in future modeling, we introduced artificial intelligence models in 3.3. Artificial neural networks differs from these traditional modeling methods by using parametric methods to build models to solve complex structural and nonlinear and dynamic problems. In section 4, we compare these models and discuss their respective advantages and limitations.


